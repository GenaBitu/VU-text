\chapter{An algebraic formalism for Multi-instance learning}

\VUname{Multi-instance learning} is a class of machine learning algorithms, first described by \cite{dietterich_solving_1997}, that can be used for both \VUname{supervised} (see \cite{amores_multiple_2013}) as well as \VUname{unsupervised} (see \cite{zhang_multi-instance_2009}, \cite{chen_contextual_2012}) learning. While classical machine learning algorithms typically require a sample to be represented as a vector of numbers, multi-instance learning relaxes this requirement by representing a sample as a \VUname{bag} of an arbitrary number of objects with a singular label for the whole bag. \cite{dedic_hierarchicke_2017} describes a novel approach to the formalization of multi-instance learning, a liberal translation of which constitutes the following chapter.
In order to correctly describe the notion of a bag using mathematical notation, a few theoretical definitions are needed.

\section{Multisets}

A multiset is a set allowing repeated elements, or alternatively an unordered tuple. \cite{knuth_art_1968} defines the concept as follows:

\begin{define}
	Let \( \VUset{A} \) be a set and \( m : \VUset{A} \to \VUfield{N} \setminus \left\{ 0 \right\} \). The tuple \( \left( \VUset{A}, m \right) \) is called a \VUname{multiset} over the set \( \VUset{A} \). For an \( a \in \VUset{A} \), the number \( m \left( a \right) \) is called the \VUname{multiplicity} (i. e. number of occurrences) of \( a \). If \( a \in \VUset{A} \), then \( a \) is called an element of \( \left( \VUset{A}, m \right) \) and denoted as
	\[ a \in \left( \VUset{A}, m \right) \]
\end{define}

Obviously, a multiset is a generalization of a set and as such it is usually written as an enumeration of the elements which are repeated as many times as is their multiplicity.

\begin{example}
	The multiset \( \left( \left\{ a, b, c \right\}, \left\{ \left( a, 4 \right), \left( b, 1 \right), \left( c, 6 \right) \right\} \right) \) may be written as \( \left\{ a, a, a, a, b, c, c, c, c, c, c \right\} \).
\end{example}

\begin{remark}
	A set can be seen as a special case of a multiset where each element has a multiplicity of 1.
\end{remark}

\begin{define}
	Let \( \VUset{A} \) be a set. A multiset \( \left( \VUset{B}, m \right) \) is called a \VUname{submultiset} of \( \VUset{A} \) if \( \VUset{B} \subset \VUset{A} \). This is denoted as
	\[ \left( \VUset{B}, m \right) \subset \VUset{A} \]
\end{define}

The previous definition allows for a generalization of the concept of a power set, which is usually denoted as \( \mathcal{P} \left( \VUset{A} \right) \) or \( 2^{\VUset{A}} \).

\begin{define}
	Let \( \VUset{A} \) be a set. A \VUname{power multiset} of \( \VUset{A} \) is the set
	\[ \mathcal{P}^M \left( \VUset{A} \right) = \left\{ \left( \VUset{B}, m \right) \middle| \left( \VUset{B}, m \right) \text{is a multiset} \wedge \left( \VUset{B}, m \right) \subset \VUset{A} \right\} \]
\end{define}

\begin{remark}
	A power set of \( \VUset{A} \) may be seen as a set of all function from \( \VUset{A} \) to \( \left\{ 0, 1 \right\} \), sometimes written as \( \left\{ 0, 1 \right\}^{\VUset{A}} \). \( \left\{ 0, 1 \right\} \) coincides with the von Neumann ordinal 2, therefore the power set is often denoted \( 2^{\VUset{A}} \). Similarly, the power multiset may be seen as a set of all functions from \( \VUset{A} \) to \( \VUfield{N} \) ( where \( 0 \in \VUfield{N} \)). It is therefore possible to use an analogous notation \( \mathcal{P}^M \left( \VUset{A} \right) = \VUfield{N}^\VUset{A} \). This analogy extends to the size of the sets in question where
	\[ \left\lvert 2^{\VUset{A}} \right\rvert = 2^{\left\lvert \VUset{A} \right\rvert} \qquad \text{and} \qquad \left\lvert \VUfield{N}^\VUset{A} \right\rvert = + \infty \]
\end{remark}

\begin{define}
	Let \( \VUset{B} = \left( \VUset{A}, m \right) \) be a multiset where \( \VUset{A} \) is finite. The cardinality of the multiset \( \VUset{B} \) is
	\[ \left\lvert \VUset{B} \right\rvert = \sum_{a \in \VUset{A}} m \left( a \right) \]
\end{define}
